%%%%%%%%%%%%%%%%%%%%%%%%%%%%%%%%%%%%%%%%%%%%%%%%%%%%%%%%%%%
\section{Разложение вектора на компоненты}
%%%%%%%%%%%%%%%%%%%%%%%%%%%%%%%%%%%%%%%%%%%%%%%%%%%%%%%%%%%
Введём в прямоугольной системе координат векторы единичной длины
$\vv{i}$, $\vv{j}$, то есть такие, что их длины равны единице:
\begin{align}
  |\vv{i}| = 1, \quad |\vv{j}| = 1.
\end{align}

Их называют \textbf{координатными векторами}.
Направляют их так, чтобы направление вектора $\vv{i}$ совпало с
направлением оси $Ox$, а направление вектора $\vv{j}$ \bdash
с направлением оси $Oy$ (рисунок~\ref{pic:vec_comp}).

Рассмотрим произвольный вектор $\vv{a}$, координаты которого в выбранной
прямоугольной системе координат равны \{$a_x; a_y$\}.

Тогда вектор $\vv{a}$ можно представить в виде суммы двух векторов:
\begin{align}
  &\vv{a} = \vv{a_x} + \vv{a_y}.
\end{align}

Векторы $\vv{a_x}$ и $\vv{a_y}$ называют \textbf{компонентами вектора} $\vv{a}$:
\begin{align}
  &\vv{a_x} = a_x\cdot \vv{i},\\
  &\vv{a_y} = a_y\cdot \vv{j}.
\end{align}

Действительно, сложим компоненты вектора $\vv{a}$ по правилу параллелограмма.
Для этого совместим их началами, достроим на них как на сторонах
параллелограмм (в данном случае он будет являться прямоугольником).
Диагональ в этом прямоугольнике совпадает с вектором $\vv{a}$.

\begin{figure}[h]
  \centering
  \begin{tikzpicture}
    \begin{axis}[
        axis lines=middle,
        axis line style = thick,
        axis equal image,
        grid=major,
        xmin=-2,
        xmax=9,
        ymin=-2,
        ymax=6,
        xlabel=$x$,
        ylabel=$y$,
        xlabel style={below right},
        ylabel style={left},
        xtick={-1,...,8},
        ytick={-1,...,5},
        yticklabels={,0,1,},
        xticklabels={,0,1,},
        %tick style={thick},
        width=12cm,
        %scale=.9,
        %transform shape
      ]
    \end{axis}
    \coordinate (A) at (3.8cm,3.8cm);
    \coordinate (B) at (8.535cm,6.64cm);
    \coordinate (C) at (1.885cm,2.86cm);
    \coordinate (D) at (2.87cm,1.88cm);
    \coordinate (OO) at (1.885cm,1.88cm);
    \coordinate [label=below:\large $0$](O) at (1.54cm,1.86cm);
    \coordinate (GG) at (8.535cm,3.8cm);
    \coordinate (H) at (3.8cm,6.64cm);
    
    \draw [-Latex, ultra thick] (A) --  node[above left,black] {\large $\vv{a}$} (B);
    \draw [-Latex, ultra thick] (A) --  node[left,black] {\large $\vv{a_y}$} (H);
    \draw [-Latex, ultra thick] (A) --  node[below,black] {\large $\vv{a_x}$} (GG);
    % coord
    \draw [-Latex, ultra thick] (OO) --  node[left,black] {$\vv{j}$} (C);
    \draw [-Latex, ultra thick] (OO) --  node[below,black] {$\vv{i}$} (D);
    % right angle
    \draw (3.8,4.1)-|(4.1,3.8);
    \draw[dashed, thick] (B) -- (GG);
    \draw[dashed, thick] (H) -- (B);
  \end{tikzpicture}
  \caption{\small Координатные векторы и разложение вектора на компоненты.}\label{pic:vec_comp}
\end{figure}

