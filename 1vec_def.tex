%%%%%%%%%%%%%%%%%%%%%%%%%%%%%%%%%%%%%%%%%%%%%%%%%%%%%%%%%%%
\section{Понятие вектора}
%%%%%%%%%%%%%%%%%%%%%%%%%%%%%%%%%%%%%%%%%%%%%%%%%%%%%%%%%%%

Точки, которые являются концами произвольного отрезка, называют
\textbf{граничными точками отрезка}.

\textbf{Вектор} \bdash отрезок, для которого указано, какая из его
граничных точек считается началом, а какая \bdash концом.

Векторы обозначают двумя заглавными буквами со стрелкой над ними, например
$\vv{AB}$. Первая буква обозначает начало вектора, вторая \bdash конец.
Векторы часто обозначают одной строчной буквой со стрелкой над ней, например,
$\vv{a}$, $\vv{b}$, $\vv{c}$. В книгах векторы обозначаются жирным прямым шрифтом: 
\textbf{a}, \textbf{b}, \textbf{c}.

\begin{figure}[h]
  \centering
  \begin{tikzpicture}[thick]
    \coordinate [label=below left:$A$] (A) at (0cm,0cm);
    \coordinate [label=above:$B$](B) at (3cm,2cm);
    
    \draw [fill=black] (A) circle (1.5pt);% node [above] {A};
    \draw [fill=black] (B) circle (1.5pt);% node [right] {B};
    
    \draw [-Latex, very thick] (A) -- (B);
    
    \draw (1.2cm,1.3cm) node {$\vv{a}$};
    
    \draw (-1.9cm,0cm) node {начало вектора};
    \draw (4.8cm,2.0cm) node {конец вектора};
  \end{tikzpicture}
  \caption{\small Вектор.}\label{pic:vec}
\end{figure}

Любая точка плоскости также является вектором. В этом случае вектор называется
\textbf{нулевым}. Начало нулевого вектора совпадает с его концом, на рисунке
такой вектор изображается точкой. Обозначается нулевой вектор так: $\vv{MM}$ или $\vv{0}$.

\textbf{Длиной} или \textbf{модулем} ненулевого \textbf{вектора} $\vv{AB}$ называется длина отрезка $AB$.
Длина вектора обозначается так: $|\vv{AB}|$, $|\vv{a}|$ или просто буквой без стрелки $a$.
Длина нулевого вектора равна нулю: $|\vv{0}| = 0$.

В физике в отличие от геометрии модули величин имеют \textbf{размерность}.
Например, модуль скорости измеряется
в метрах в секунду (м/с), модуль силы \bdash в ньютонах (Н), модуль напряжённости электрического
поля \bdash в вольтах на метр (В/м), модуль индукции магнитного поля \bdash в теслах (Тл) и т.д.
