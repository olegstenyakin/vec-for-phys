%%%%%%%%%%%%%%%%%%%%%%%%%%%%%%%%%%%%%%%%%%%%%%%%%%%%%%%%%%%
\section{Координаты (проекции) векторов}
%%%%%%%%%%%%%%%%%%%%%%%%%%%%%%%%%%%%%%%%%%%%%%%%%%%%%%%%%%%

%%%%%%%%%%%%%%%%%%%%%%%%%%%%%%%%%%%%%%%%%%%%%%%%%%%%%%%%%%%
\subsection{Связь между координатами вектора и координатами
его начала и конца}
%%%%%%%%%%%%%%%%%%%%%%%%%%%%%%%%%%%%%%%%%%%%%%%%%%%%%%%%%%%
Рассмотрим произвольный вектор $\vv{a}$ в прямоугольной системе координат.
Опустив перпендикуляры из начала и конца вектора на оси системы координат,
получим координаты начала и конца вектора: ($x_1, y_1$) и ($x_2, y_2$)
соответственно.

Говорят, что \textbf{каждая координата вектора равна разности соответствующих
координат его конца и начала}.
Вектор $\vv{a}$ имеет координаты {\large\{$x_2-x_1; y_2-y_1$\}}.

Координаты вектора обозначают той же буквой, что и сам вектор, но без стрелки над ней
и с нижним индексом, указывающим ось координат. Координаты вектора $\vv{a}$ можно записать
как {\large\{$a_x; a_y$\}}.

На рисунке~\ref{pic:vec_coord} определим координаты вектора $\vv{a}$:
\begin{align*}
  &a_x = x_2-x_1 = 7 -1 = 6,\\
  &a_y = y_2-y_1 = 4-1=3.
\end{align*}
То есть координаты вектора $\vv{a}$ \{$6; 3$\}.

\begin{figure}[h]
  \centering
  \begin{tikzpicture}
    \begin{axis}[
        axis lines=middle,
        axis line style = thick,
        axis equal image,
        grid=major,
        xmin=-2,
        xmax=9,
        ymin=-2,
        ymax=6,
        xlabel=$x$,
        ylabel=$y$,
        xlabel style={below right},
        ylabel style={left},
        xtick={-1,...,8},
        ytick={-1,...,5},
        yticklabels={,,},
        xticklabels={,,},
        %axis on top,
        %tick style={thick},
        width=12cm
      ]
    \end{axis}
    \coordinate (A) at (2.84cm,2.84cm);
    \coordinate (B) at (8.535cm,5.685cm);
    \coordinate [label=below:\large $x_1$](C) at (2.84cm,1.8cm);
    \coordinate [label=below:\large $x_2$](D) at (8.535cm,1.887cm);
    \coordinate [label=below:\large $0$](O) at (1.54cm,1.86cm);
    \coordinate (CC) at (2.84cm,1.2cm);
    \coordinate (DD) at (8.535cm,1.2cm);
    \coordinate [label=left:\large $y_1$](E) at (1.8cm,2.84cm);
    \coordinate [label=left:\large $y_2$](F) at (1.8cm,5.685cm);
    \coordinate (EE) at (1.1cm,2.84cm);
    \coordinate (FF) at (1.1cm,5.685cm);
    
    \draw [-Latex, ultra thick] (A) --  node[above left,black] {\large $\vv{a}$} (B);

    \draw[dashed, thick] (A) -- (C);
    \draw[dashed, thick] (A) -- (E);
    \draw[dashed, thick] (B) -- (D);
    \draw[dashed, thick] (B) -- (F);
    \draw [decorate, decoration={brace, mirror, amplitude=10pt}] (CC) -- (DD);
    \draw [left, decorate, decoration={brace, amplitude=10pt}] (EE) -- (FF);
    \draw (5.6cm,0.4cm) node {\large $a_x = x_2 - x_1$};
    \draw (-0.8cm,4.3cm) node {\large $a_y = y_2 - y_1$};
  \end{tikzpicture}
  \caption{\small Координаты вектора.}\label{pic:vec_coord}
\end{figure}

\textbf{В физике координаты векторных величин называют проекциями}.

Например, $v_x$ \bdash проекция вектора скорости $\vv{v}$ на ось $Ox$.

\textbf{Значения проекций векторов могут быть отрицательными, положительными или равными нулю}.

%%%%%%%%%%%%%%%%%%%%%%%%%%%%%%%%%%%%%%%%%%%%%%%%%%%%%%%%%%%
\subsection{Связь между координатами вектора, его модулем
и углом между вектором и одной из осей координат}
%%%%%%%%%%%%%%%%%%%%%%%%%%%%%%%%%%%%%%%%%%%%%%%%%%%%%%%%%%%
На практике \textbf{при решении задач по физике}
возможно определить координаты начала и конца
лишь для вектора перемещения. Во всех остальных случаях очень удобно
опрелять проекции векторов, зная их модуль и угол, который вектор
составляет с одной из осей координат.


%%%%%%%%%%%%%%%%%%%%%%%%%%%%%%%%%%%%%%%%%%%%%%%%%%%%%%%%%%%
\subsubsection{Вектор параллелен или перпендикулярен оси координат}
%%%%%%%%%%%%%%%%%%%%%%%%%%%%%%%%%%%%%%%%%%%%%%%%%%%%%%%%%%%
Рассмотрим векторы $\vv{a}$ и $\vv{g}$, длины которых известны и равны $a$ и $g$
соответственно, в прямоугольной системе координат (рисунок~\ref{pic:vec_coord0}).
Пусть вектор $\vv{a}$ параллелен оси $Ox$
и его направление совпадает с направлением оси $Ox$,
вектор $\vv{g}$ параллелен оси $Oy$
и его направление противоположно направлению оси $Oy$.

\begin{figure}[h]
  \centering
  \begin{tikzpicture}
    \begin{axis}[
        axis lines=middle,
        axis line style = thick,
        axis equal image,
        grid=major,
        xmin=-2,
        xmax=9,
        ymin=-2,
        ymax=6,
        xlabel=$x$,
        ylabel=$y$,
        xlabel style={below right},
        ylabel style={left},
        xtick={-1,...,8},
        ytick={-1,...,5},
        yticklabels={,,},
        xticklabels={,,},
        %tick style={thick},
        width=12cm
      ]
    \end{axis}
    % a
    \coordinate (A) at (2.84cm,2.84cm);
    \coordinate (B) at (6.63cm,2.84cm);
    \coordinate [label=below:$x_1$](C) at (2.84cm,1.8cm);
    \coordinate [label=below:$x_2$](D) at (6.63cm,1.8cm);
    \coordinate [label=below:\large $0$](O) at (1.54cm,1.86cm);
    \coordinate (CC) at (2.84cm,1.2cm);
    \coordinate (DD) at (6.63cm,1.2cm);
    \coordinate (E) at (1.8cm,2.84cm);
    % g
    \coordinate [label=left:$y_1$](CCC) at (1.8cm,6.63cm);
    \coordinate [label=left:$y_2$](DDD) at (1.8cm,3.78cm);
    \coordinate (FF) at (1.1cm,6.63cm);
    \coordinate (EE) at (1.1cm,3.78cm);
    \coordinate (F) at (8.535cm,1.88cm);
    \coordinate (FFF) at (8.535cm,6.63cm);
    \coordinate (EEE) at (8.535cm,3.78cm);
    %a
    \draw [-Latex, ultra thick] (A) --  node[above,black] {\large $\vv{a}$} (B);
    %g
    \draw [-Latex, ultra thick] (FFF) --  node[right,black] {\large $\vv{g}$} (EEE);
    
    \draw[dashed, thick] (A) -- (C);
    \draw[dashed, thick] (E) -- (A);
    \draw[dashed, thick] (CCC) -- (FFF);
    \draw[dashed, thick] (DDD) -- (EEE);
    
    % right angle
    \draw (1.9,2.55)-|(2.2,2.85);
    \draw (8.235,1.89)|-(8.535,2.19);

    \draw[dashed, thick] (B) -- (D);
    \draw[dashed, thick] (EEE) -- (F);
    \draw [decorate, decoration={brace, mirror, amplitude=10pt}] (CC) -- (DD);
    \draw [left, decorate, decoration={brace, amplitude=10pt}] (EE) -- (FF);
    \draw (5cm,0.5cm) node {$\hly{a_x=x_2 - x_1 = a}$};
    \draw (1cm,2.8cm) node {$\hly{a_y=0}$};
    \draw (8.55cm,1.4cm) node {$\hlt{g_x=0}$};
    \draw (-1.2cm,5.2cm) node {$\hlt{g_y=y_2-y_1 = -g}$};
  \end{tikzpicture}
  \caption{\small Определение проекций векторов, параллельного и перпендикулярного
    оси координат.}\label{pic:vec_coord0}
\end{figure}

Для определения координат вектора $\vv{a}$ опустим перпендикуляры из начала и конца
вектора на оси координат и определим разности координат начала и конца вектора.
Обратим внимание, что разность координат $x_2-x_1$ по оси $Ox$ в точности
равна длине $a$ вектора. А разность координат по оси $Oy$ равна нулю, так как
координаты начала и конца вектора $\vv{a}$ по этой оси совпадают.
Итак:
\begin{align*}
  &a_x = a,\\
  &a_y = 0.
\end{align*}

Аналогично для определения координат вектора $\vv{g}$ опустим перпендикуляры из начала и конца
вектора на оси координат и определим разности координат начала и конца вектора.
Модуль разности координат $|y_2-y_1|$ по оси $Oy$ в точности
равна длине $g$ вектора. У вектора $\vv{g}$ координата конца меньше,
чем координата начала $y_2<y_1$, поэтому координата $g_y=y_2-y_1$ меньше нуля.
А разность координат по оси $Ox$ равна нулю, так как
координаты начала и конца вектора $\vv{g}$ по этой оси совпадают.
Получаем:
\begin{align*}
  &g_x  = 0,\\
  &g_y = -g.
\end{align*}

Сделаем \textbf{важный вывод}:
если направление вектора и оси координат совпадают, то его координата
(проекция) на эту ось равна модулю вектора ($a_x = a$).
Если же направление вектора и оси координат противоположны, то его координата
на эту ось равна модулю вектора, взятому со знаком минус ($g_y = -g$).
Если вектор перпендикулярен данной оси координат, то его проекция на эту равна 0.

Запишем \textbf{2 шага по определению значения проекции вектора}:
\begin{itemize}
\item определяем знак проекции. Если начало вектора расположено ближе к началу
  координат, проекция больше нуля. Если наоборот, то меньше нуля.
  \item определяем модуль проекции.
\end{itemize}

%%%%%%%%%%%%%%%%%%%%%%%%%%%%%%%%%%%%%%%%%%%%%%%%%%%%%%%%%%%
\subsubsection{Вектор направлен под произвольным углом к оси координат}
%%%%%%%%%%%%%%%%%%%%%%%%%%%%%%%%%%%%%%%%%%%%%%%%%%%%%%%%%%%
%%%%%%%%%%%%%%%%%%%%%%%%%%%%%%%%%%%%%%%%%%%%%%%%%%%%%%%%%%%
\textbf{1 Положительные проекции}\\
%%%%%%%%%%%%%%%%%%%%%%%%%%%%%%%%%%%%%%%%%%%%%%%%%%%%%%%%%%%
Рассмотрим вектор $\vv{a}$, длина которого известна и равна $a$,
в прямоугольной системе координат (рисунок~\ref{pic:vec_coord1}).
Пусть угол между вектором $\vv{a}$ и горизонталью известен и равен $\upalpha$.
Под таким углом вектор направлен к оси $Ox$, а также к любой другой прямой,
параллельной оси $Ox$.

Для определения его координат опустим перпендикуляры из начала и конца вектора
на оси координат. Продлим перпендикуляр из начала вектора на ось $Oy$ до его пересечения
с перпендикуляром из конца на ось $Ox$ в точке $B$.

В получившимся прямоугольном треугольнике $ABC$ гипотенуза $AC = a$, а угол $\angle A = \upalpha$.
Обратим внимание на катеты в этом треугольнике.
Катет $AB$ в точности равен разности координат на ось $Ox$ конца и начала вектора $\vv{a}$, то есть
$AB = a_x = x_2-x_1$. А катет $BC$ равен разности координат
конца и начала вектора на ось $Oy$: $BC = a_y = y_2-y_1$.

Вспомним соотношения в прямоугольном треугольнике:
\begin{itemize}
\item \textbf{синус} \bdash отношение противолежащего к углу катета к гипотенузе,
\item \textbf{косинус} \bdash отношение прилежащего к углу катета к гипотенузе,
\item \textbf{тангенс} \bdash отношение противолежащего к углу катета к прилежащему.
\end{itemize}

% положительные проекции
\begin{figure}[h]
  \centering
  \begin{tikzpicture}
    \begin{axis}[
        axis lines=middle,
        axis line style = thick,
        axis equal image,
        grid=major,
        xmin=-2,
        xmax=9,
        ymin=-2,
        ymax=6,
        xlabel=$x$,
        ylabel=$y$,
        xlabel style={below right},
        ylabel style={left},
        xtick={-1,...,8},
        ytick={-1,...,5},
        yticklabels={,,},
        xticklabels={,,},
        %tick style={thick},
        width=12cm
      ]
    \end{axis}
    \coordinate [label=above left:\large $A$](A) at (2.84cm,2.84cm);
    \coordinate [label=above right:\large $C$](B) at (8.535cm,5.685cm);
    \coordinate [label=below:\large $x_1$](C) at (2.84cm,1.8cm);
    \coordinate [label=below:\large $x_2$](D) at (8.535cm,1.887cm);
    \coordinate [label=below:\large $0$](O) at (1.54cm,1.86cm);
    \coordinate (CC) at (2.84cm,1.2cm);
    \coordinate (DD) at (8.535cm,1.2cm);
    \coordinate [label=left:\large $y_1$](E) at (1.8cm,2.84cm);
    \coordinate [label=left:\large $y_2$](F) at (1.8cm,5.685cm);
    \coordinate (EE) at (1.1cm,2.84cm);
    \coordinate (FF) at (1.1cm,5.685cm);
    \coordinate [label=above right:\large $B$](GG) at (8.535cm,2.84cm);
    
    \draw [-Latex, ultra thick] (A) --  node[above left,black] {\large $\vv{a}$} (B);

    \draw[dashed, thick] (A) -- (C);
    \draw[dashed, thick] (E) -- (GG);
    % right angle
    \draw (8.235,2.84)|-(8.535,3.14);
    % angle
    \tkzMarkAngle[arc=l, size=1cm, mark=none](GG,A,B)
    \tkzLabelAngle[pos = 1.4](GG,A,B){$\upalpha$}
    \draw[dashed, thick] (B) -- (D);
    \draw[dashed, thick] (B) -- (F);
    \draw [decorate, decoration={brace, mirror, amplitude=10pt}] (CC) -- (DD);
    \draw [left, decorate, decoration={brace, amplitude=10pt}] (EE) -- (FF);
    \draw (5.4cm,2.4cm) node {\large $\hly{a_x = a\cos{\upalpha}}$};
    \draw (10.1cm,4.1cm) node {\large $\hly{a_y = a\sin{\upalpha}}$};
    \draw (5.6cm,0.5cm) node {\large $a_x>0$};
    \draw (0cm,4.2cm) node {\large $a_y>0$};
  \end{tikzpicture}
  \caption{\small Определение проекций вектора по его модулю
    и углу к одной из осей координат.
    Вектор с положительными проекциями по обеим осям координат.}\label{pic:vec_coord1}
\end{figure}

В нашем примере:
\begin{align}
  \hly{\sin{\upalpha} = \dfrac{a_y}{a},\quad \cos{\upalpha} = \dfrac{a_x}{a},\quad \tg{\upalpha} = \dfrac{a_y}{a_x}.}
\end{align}

Таким образом, координаты вектора $\vv{a}$ выражаются через его модуль
и угол между вектором $\vv{a}$ и горизонталью через отношения в прямоугольном треугольнике:
\begin{align}
  &\hly{a_x = a\cos{\upalpha}},\\
  &\hly{a_y = a\sin{\upalpha}}.
\end{align}

Обратим внимание, что проекции на обе оси координат положительны, так как 
начало вектора расположено ближе к началу координат для обеих осей.

\clearpage

%%%%%%%%%%%%%%%%%%%%%%%%%%%%%%%%%%%%%%%%%%%%%%%%%%%%%%%%%%%
\textbf{2 Отрицательные проекции}\\
%%%%%%%%%%%%%%%%%%%%%%%%%%%%%%%%%%%%%%%%%%%%%%%%%%%%%%%%%%%
Рассмотрим случай, при котором координата начала вектора расположена
дальше от начала координат, чем координата конца,
и по оси $Ox$, и по оси $Oy$ (рисунок~\ref{pic:vec_coord2}).
В этом случае, обе проекции вектора будут отрицательны. А их модули можно
снова определить из прямоугольного треугольника через синус и косинус угла,
под которым вектор направлен к горизонтали:
\begin{align}
  &a_x = -a\cos{\upalpha},\\
  &a_y = -a\sin{\upalpha}.
\end{align}

% отрицательные проекции
\begin{figure}[h]
  \centering
  \begin{tikzpicture}
    \begin{axis}[
        axis lines=middle,
        axis line style = thick,
        axis equal image,
        grid=major,
        xmin=-2,
        xmax=9,
        ymin=-2,
        ymax=6,
        xlabel=$x$,
        ylabel=$y$,
        xlabel style={below right},
        ylabel style={left},
        xtick={-1,...,8},
        ytick={-1,...,5},
        yticklabels={,,},
        xticklabels={,,},
        %tick style={thick},
        width=12cm
      ]
    \end{axis}
    \coordinate [label=above left:\large $A$](A) at (2.84cm,2.84cm);
    \coordinate [label=above right:\large $C$](B) at (8.535cm,5.685cm);
    \coordinate [label=below:\large $x_2$](C) at (2.84cm,1.8cm);
    \coordinate [label=below:\large $x_1$](D) at (8.535cm,1.887cm);
    \coordinate [label=below:\large $0$](O) at (1.54cm,1.86cm);
    \coordinate (CC) at (2.84cm,1.2cm);
    \coordinate (DD) at (8.535cm,1.2cm);
    \coordinate [label=left:\large $y_2$](E) at (1.8cm,2.84cm);
    \coordinate [label=left:\large $y_1$](F) at (1.8cm,5.685cm);
    \coordinate (EE) at (1.1cm,2.84cm);
    \coordinate (FF) at (1.1cm,5.685cm);
    \coordinate [label=above right:\large $B$](GG) at (8.535cm,2.84cm);
    
    \draw [-Latex, ultra thick] (B) --  node[above left,black] {\large $\vv{a}$} (A);

    \draw[dashed, thick] (A) -- (C);
    \draw[dashed, thick] (E) -- (GG);
    % right angle
    \draw (8.235,2.84)|-(8.535,3.14);
    % angle
    \tkzMarkAngle[arc=l, size=1cm, mark=none](GG,A,B)
    \tkzLabelAngle[pos = 1.4](GG,A,B){$\upalpha$}
    \draw[dashed, thick] (B) -- (D);
    \draw[dashed, thick] (B) -- (F);
    \draw [decorate, decoration={brace, mirror, amplitude=10pt}] (CC) -- (DD);
    \draw [left, decorate, decoration={brace, amplitude=10pt}] (EE) -- (FF);
    \draw (5.4cm,2.4cm) node {\large $\hly{a_x = -a\cos{\upalpha}}$};
    \draw (10.2cm,4.1cm) node {\large $\hly{a_y = -a\sin{\upalpha}}$};
    \draw (5.6cm,0.5cm) node {\large $a_x<0$};
    \draw (0cm,4.2cm) node {\large $a_y<0$};
  \end{tikzpicture}
  \caption{\small Вектор с отрицательными проекциями по обеим осям координат.}\label{pic:vec_coord2}
\end{figure}

\clearpage

%%%%%%%%%%%%%%%%%%%%%%%%%%%%%%%%%%%%%%%%%%%%%%%%%%%%%%%%%%%
\textbf{3 Проекции разных знаков}\\
%%%%%%%%%%%%%%%%%%%%%%%%%%%%%%%%%%%%%%%%%%%%%%%%%%%%%%%%%%%
На рисунке~\ref{pic:vec_coord3} изображён вектор с проекциями разных знаков:
с отрицательной проекцией по оси $Ox$ и положительной по оси $Oy$.
Кроме того, в данном случае задан угол между вектором и вертикалью, поэтому
в данном случае проекции вектора $\vv{a}$ будут следующие:
\begin{align}
  &a_x = -a\sin{\upalpha},\\
  &a_y = a\cos{\upalpha}.
\end{align}

% проекции разных знаков
\begin{figure}[h]
  \centering
  \begin{tikzpicture}
    \begin{axis}[
        axis lines=middle,
        axis line style = thick,
        axis equal image,
        grid=major,
        xmin=-2,
        xmax=9,
        ymin=-2,
        ymax=6,
        xlabel=$x$,
        ylabel=$y$,
        xlabel style={below right},
        ylabel style={left},
        xtick={-1,...,8},
        ytick={-1,...,5},
        yticklabels={,,},
        xticklabels={,,},
        %tick style={thick},
        width=12cm
      ]
    \end{axis}
    \coordinate (A) at (2.84cm,2.84cm);
    \coordinate (B) at (8.535cm,5.685cm);
    \coordinate [label=below:\large $x_2$](C) at (2.84cm,1.8cm);
    \coordinate [label=below:\large $x_1$](D) at (8.535cm,1.887cm);
    \coordinate [label=below:\large $0$](O) at (1.54cm,1.86cm);
    \coordinate (CC) at (2.84cm,1.2cm);
    \coordinate (DD) at (8.535cm,1.2cm);
    \coordinate [label=left:\large $y_1$](E) at (1.8cm,2.84cm);
    \coordinate [label=left:\large $y_2$](F) at (1.8cm,5.685cm);
    \coordinate (EE) at (1.1cm,2.84cm);
    \coordinate (FF) at (1.1cm,5.685cm);
    \coordinate (GG) at (8.535cm,2.84cm);
    \coordinate (H) at (2.84cm,5.685cm);
    
    \draw [-Latex, ultra thick] (GG) --  node[above right,black] {\large $\vv{a}$} (H);

    \draw[dashed, thick] (H) -- (C);
    \draw[dashed, thick] (E) -- (GG);
    %\draw[dashed, thick] (E) -- (GG);
    % right angle
    \draw (2.84,3.14)-|(3.14,2.84);
    \draw (8.235,5.685)|-(8.535,5.385);
    % angle
    \tkzMarkAngle[arc=l, size=0.8cm, mark=none](B,GG,H)
    \tkzLabelAngle[pos = 1.2](B,GG,H){$\upalpha$}
    \tkzMarkAngle[arc=l, size=0.8cm, mark=none](A,H,GG)
    \tkzLabelAngle[pos = 1.2](A,H,GG){$\upalpha$}
    \draw[dashed, thick] (B) -- (D);
    \draw[dashed, thick] (B) -- (F);
    \draw [decorate, decoration={brace, mirror, amplitude=10pt}] (CC) -- (DD);
    \draw [left, decorate, decoration={brace, amplitude=10pt}] (EE) -- (FF);
    \draw (5.4cm,2.4cm) node {\large $\hly{a_x = -a\sin{\upalpha}}$};
    \draw (10.2cm,4.1cm) node {\large $\hly{a_y = a\cos{\upalpha}}$};
    \draw (5.6cm,0.5cm) node {\large $a_x<0$};
    \draw (0cm,4.2cm) node {\large $a_y>0$};
  \end{tikzpicture}
  \caption{\small Вектор с отрицательной проекцией по оси $Ox$
    и положительной по оси $Oy$.}\label{pic:vec_coord3}
\end{figure}

\clearpage

%%%%%%%%%%%%%%%%%%%%%%%%%%%%%%%%%%%%%%%%%%%%%%%%%%%%%%%%%%%
\subsection{Проекция суммы векторов и произведения вектора на число}
%%%%%%%%%%%%%%%%%%%%%%%%%%%%%%%%%%%%%%%%%%%%%%%%%%%%%%%%%%%

Часто в задачах по физике встречаются векторные уравнения, в которые
входят суммы векторов: закон сложения скоростей, второй закон Ньютона,
закон сохранения импульса и так далее. Отметим важные свойства проекций
векторов, которые используются в таких задачах.

%%%%%%%%%%%%%%%%%%%%%%%%%%%%%%%%%%%%%%%%%%%%%%%%%%%%%%%%%%%
\textbf{Проекция суммы векторов}\\
%%%%%%%%%%%%%%%%%%%%%%%%%%%%%%%%%%%%%%%%%%%%%%%%%%%%%%%%%%%
Рассмотрим два вектора $\vv{a}$ и $\vv{b}$ в прямоугольной системе координат
(рисунок~\ref{pic:vec_sum_proj}). Сложим их по правилу треугольника и проведём
сумму $\vv{c} = \vv{a}+\vv{b}$. Покажем, что проекция вектора $\vv{c}$ на ось $Ox$
равна сумме проекций векторов $\vv{a}$ и $\vv{b}$ на эту ось.

Действительно, по определению проекция вектора $\vv{c}$ равна $c_x = x_3 - x_1$. Добавим
в правую часть этого равенства число 0, равенство от этого не изменится. Представим этот ноль
в таком виде $0 = x_2 - x_2$. Затем перегруппируем слагаемые:
\begin{align}
&c_x = x_3 - x_1 = x_3 + (x_2 - x_2) - x_1 = (x_3 - x_2) + (x_2 - x_1).
\end{align}

Выражения в скобках и есть проекции векторов $\vv{a}$ и $\vv{b}$. Итак,
\textbf{проекция суммы векторов равна сумме их проекций}:
{\large
\begin{align}
  \hly{c_x = a_x + b_x.}
\end{align}
}

\begin{figure}[h]
  \centering
  \begin{tikzpicture}
    \begin{axis}[
        axis lines=middle,
        axis line style = thick,
        axis equal image,
        grid=major,
        xmin=-2,
        xmax=9,
        ymin=-2,
        ymax=6,
        xlabel=$x$,
        ylabel=$y$,
        xlabel style={below right},
        ylabel style={left},
        xtick={-1,...,8},
        ytick={-1,...,5},
        yticklabels={,,},
        xticklabels={,,},
        %axis on top,
        %tick style={thick},
        width=12cm
      ]
    \end{axis}
    \coordinate (A) at (2.84cm,2.84cm);
    \coordinate (B) at (5.7cm,5.685cm);%4.745
    \coordinate [label=below:$x_1$](C) at (2.84cm,1.88cm);
    \coordinate [label=below:$x_2$](D) at (5.7cm,1.88cm);
    \coordinate [label=below:$x_3$](DDD) at (9.48cm,1.88cm);
    \coordinate [label=below:\large $0$](O) at (1.54cm,1.86cm);
    \coordinate (E) at (9.48cm,5.685cm);
    \coordinate (CC) at (2.84cm,1.2cm);
    \coordinate (DD) at (5.7cm,1.2cm);
    \coordinate (EE) at (9.48cm,1.2cm);
    %\coordinate (FF) at (1.1cm,5.685cm);
    
    \draw [-Latex, very thick] (A) --  node[above left,black] {\large $\vv{a}$} (B);
    \draw [-Latex, very thick] (B) --  node[above,black] {\large $\vv{b}$} (E);
    \draw [-Latex, very thick, red] (A) --  node[below right,red] {\large $\vv{c} = \vv{a}+\vv{b}$} (E);

    \draw[dashed, thick] (A) -- (C);
    \draw[dashed, thick] (D) -- (B);
    \draw[dashed, thick] (E) -- (DDD);
    \draw [decorate, decoration={brace, mirror, amplitude=10pt}] (CC) -- (EE);
    \draw (4.2cm,1.47cm) node {\footnotesize $a_x=x_2 - x_1$};%1.5
    \draw (7.6cm,1.47cm) node {\footnotesize $b_x=x_3-x_2$};
    \draw (6cm,0.4cm) node {\large $\hly{c_x = a_x + b_x}$};
  \end{tikzpicture}
  \caption{\small Проекция суммы векторов.}\label{pic:vec_sum_proj}
\end{figure}

\clearpage

%%%%%%%%%%%%%%%%%%%%%%%%%%%%%%%%%%%%%%%%%%%%%%%%%%%%%%%%%%%
\textbf{Проекция произведения вектора на число}\\
%%%%%%%%%%%%%%%%%%%%%%%%%%%%%%%%%%%%%%%%%%%%%%%%%%%%%%%%%%%
Покажем, что \textbf{проекция вектора, умноженного на число, равна
произведению проекции вектора на это число}.

Рассмотрим вектор $\vv{a}$ в прямоугольной системе координат (рисунок~\ref{pic:vec_scal_proj}).
Пусть он направлен под углом $\upalpha$ к оси $Ox$. Для удобства допустим, что $\upalpha$ острый.
Как мы показали ранее, проекция вектора $\vv{a}$ на ось $Ox$ равна $a_x = a\cos{\upalpha}$.

Умножим вектор $\vv{a}$ на произвольное число $C$. Пусть $C>1$.
Вектор удлинится в $C$ раз и сохранит прежнее направление.
Модуль получившегося в результате вектора $\vv{b} = C\cdot \vv{a}$
станет равен $C\cdot a$. 
Это значит, что проекция такого вектора на ось $Ox$ равна $b_x = C\cdot a\cos{\upalpha}$
или окончательно:
{\large
\begin{align}
  \hly{b_x = C\cdot a_x.}
\end{align}
}

\begin{figure}[h]
  \centering
  \begin{tikzpicture}
    \begin{axis}[
        axis lines=middle,
        axis line style = thick,
        axis equal image,
        grid=major,
        xmin=-2,
        xmax=15,
        ymin=-2,
        ymax=8,
        xlabel=$x$,
        ylabel=$y$,
        xlabel style={below right},
        ylabel style={left},
        xtick={-1,...,14},
        ytick={-1,...,7},
        yticklabels={,,},
        xticklabels={,,},
        %tick style={thick},
        width=0.9\textwidth
      ]
    \end{axis}
    % a
    \coordinate (A) at (2.5cm,2.5cm);
    \coordinate (B) at (5.03cm,5.03cm);
    \coordinate [label=below:\large $x_1$](C) at (2.5cm,1.7cm);
    \coordinate [label=below:\large $x_2$](D) at (5.03cm,1.7cm);
    \coordinate [label=below:\large $0$](O) at (1.4cm,1.7cm);
    \coordinate (CC) at (2.5cm,1.2cm);
    \coordinate (GG) at (5.03cm,2.5cm);
    
    \draw [-Latex, ultra thick] (A) --  node[above left,black] {\large $\vv{a}$} (B);

    \draw[dashed, thick] (A) -- (C);
    \draw[dashed, thick] (A) -- (GG);

    % angle
    \tkzMarkAngle[arc=l, size=1cm, mark=none](GG,A,B)
    \tkzLabelAngle[pos = 1.4](GG,A,B){$\upalpha$}
    \draw[dashed, thick] (B) -- (D);
    \draw (3.76cm,1.2cm) node {\large $\hly{a_x}$};
    % C a
    \coordinate (AC) at (6.7cm,2.5cm);
    \coordinate (BC) at (11.75cm,7.55cm);
    \coordinate [label=below:\large $x_3$](CCC) at (6.7cm,1.7cm);
    \coordinate [label=below:\large $x_4$](DC) at (11.75cm,1.7cm);
    \coordinate (GGG) at (11.75cm,2.5cm);
    \draw [-Latex, ultra thick] (AC) --  node[above left,black] {\large $\vv{b}= C\cdot \vv{a}$} (BC);
    \draw[dashed, thick] (AC) -- (CCC);
    \draw[dashed, thick] (AC) -- (GGG);
    \draw[dashed, thick] (BC) -- (DC);
    % angle
    \tkzMarkAngle[arc=l, size=1cm, mark=none](GGG,AC,BC)
    \tkzLabelAngle[pos = 1.4](GGG,AC,BC){$\upalpha$}
    \draw (9.2cm,1.2cm) node {\large $\hly{b_x = C\cdot a_x}$};
  \end{tikzpicture}
  \caption{\small Проекция вектора, умноженного на число.}\label{pic:vec_scal_proj}
\end{figure}

Можно показать, что данное утверждение справедливо для произвольного угла наклона $\upalpha$ вектора
к оси координат, а также для произвольной константы $C$.

\clearpage

%%%%%%%%%%%%%%%%%%%%%%%%%%%%%%%%%%%%%%%%%%%%%%%%%%%%%%%%%%%
\subsection{Вычисление длины вектора по его координатам}
%%%%%%%%%%%%%%%%%%%%%%%%%%%%%%%%%%%%%%%%%%%%%%%%%%%%%%%%%%%

Важнейшим следствием связи координат вектора и его модуля
является возможность выполнить обратную операцию:
вычислить длину вектора по его координатам.

Длину вектора $\vv{a}$, координаты которого в выбранной
прямоугольной системе координат равны \{$a_x; a_y$\}, можно
найти \textbf{по теореме Пифагора} (рисунок~\ref{pic:vec_dl}):
\begin{align}
  &\hly{a = \sqrt{a_x^2 + a_y^2}}.
\end{align}

Так как проеции вектора равны разности координат его конца и начала
$a_x = x_2 - x_1$, $a_y = y_2 - y_1$, то:
\begin{align}
  &\hly{a = \sqrt{(x_2 - x_1)^2 + (y_2 - y_1)^2}}.
\end{align}

\begin{figure}[h]
  \centering
  \begin{tikzpicture}
    \begin{axis}[
        axis lines=middle,
        axis line style = thick,
        axis equal image,
        grid=major,
        xmin=-2,
        xmax=9,
        ymin=-2,
        ymax=6,
        xlabel=$x$,
        ylabel=$y$,
        xlabel style={below right},
        ylabel style={left},
        xtick={-1,...,8},
        ytick={-1,...,5},
        yticklabels={,,},
        xticklabels={,,},
        %tick style={thick},
        width=12cm
      ]
    \end{axis}
    \coordinate (A) at (2.84cm,2.84cm);
    \coordinate (B) at (8.535cm,5.685cm);
    \coordinate [label=below:\large $x_1$](C) at (2.84cm,1.8cm);
    \coordinate [label=below:\large $x_2$](D) at (8.535cm,1.887cm);
    \coordinate [label=below:\large $0$](O) at (1.54cm,1.86cm);
    \coordinate (CC) at (2.84cm,1.2cm);
    \coordinate (DD) at (8.535cm,1.2cm);
    \coordinate [label=left:\large $y_1$](E) at (1.8cm,2.84cm);
    \coordinate [label=left:\large $y_2$](F) at (1.8cm,5.685cm);
    \coordinate (EE) at (1.1cm,2.84cm);
    \coordinate (FF) at (1.1cm,5.685cm);
    \coordinate (GG) at (8.535cm,2.84cm);
    
    \draw [-Latex, ultra thick] (A) --  node[above left,black] {\large $\vv{a}$} (B);

    \draw[dashed, thick] (A) -- (C);
    \draw[dashed, thick] (E) -- (GG);
    % right angle
    \draw (8.235,2.84)|-(8.535,3.14);

    \draw[dashed, thick] (B) -- (D);
    \draw[dashed, thick] (B) -- (F);
    \draw (5.8cm,2.4cm) node {\large $a_x$};
    \draw (9.1cm,4.1cm) node {\large $a_y$};
  \end{tikzpicture}
  \caption{\small Вычисление длины вектора по его координатам.}\label{pic:vec_dl}
\end{figure}

