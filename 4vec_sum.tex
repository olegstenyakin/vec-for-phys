%%%%%%%%%%%%%%%%%%%%%%%%%%%%%%%%%%%%%%%%%%%%%%%%%%%%%%%%%%%
\section{Сложение векторов}
%%%%%%%%%%%%%%%%%%%%%%%%%%%%%%%%%%%%%%%%%%%%%%%%%%%%%%%%%%%
\subsection{Правило треугольника}
%%%%%%%%%%%%%%%%%%%%%%%%%%%%%%%%%%%%%%%%%%%%%%%%%%%%%%%%%%%
Пусть $\vv{a}$ и $\vv{b}$ \bdash два вектора. Отметим произвольную точку $A$
и отложим от этой точки вектор $\vv{AB}$, равный $\vv{a}$. Затем от точки $B$
(конец вектора $\vv{AB}$) отложим вектор $\vv{BC}$, равный $\vv{b}$.
Вектор $\vv{AC}=\vv{c}$ называется \textbf{суммой векторов} $\vv{a}$ и $\vv{b}$.

Таким образом, при построении суммы двух векторов начало второго вектора
примыкает с концу первого, и сумма замыкает образуемый ими треугольник.
\begin{figure}[h]
  \centering
  \begin{tikzpicture}[thick]
    \coordinate [label=below left:$A$] (A) at (0cm,0cm);
    \coordinate [label=above:$B$] (B) at (3cm,3cm);
    \coordinate [label=above:$C$](C) at (7cm,3cm);
    \coordinate (D) at (-6cm,0cm);
    \coordinate (E) at (-3cm,3cm);
    \coordinate (F) at (-5cm,4cm);
    \coordinate (G) at (-1cm,4cm);
    
    \draw [blue, -Latex, very thick] (A) -- node[above left,black] {\large $\vv{a}$} (B);
    \draw [blue, -Latex, very thick] (B) -- node[above,black] {\large $\vv{b}$} (C);
    \draw [red, -Latex, very thick] (A) -- node[below right,black] {\large $\vv{c} =\vv{a}+\vv{b}$} (C);
    \draw [blue, -Latex, very thick] (D) -- node[above left,black] {\large $\vv{a}$} (E);
    \draw [blue, -Latex, very thick] (F) -- node[above,black] {\large $\vv{b}$} (G);
    
    \draw [fill=black] (A) circle (1.5pt);% node [above] {A};
    \draw [fill=black] (B) circle (1.5pt);% node [right] {B};
    \draw [fill=black] (C) circle (1.5pt);
  \end{tikzpicture}
  \caption{\small Сложение векторов по правилу треугольника.}\label{pic:sum_tr}
\end{figure}

По правилу треугольника \textbf{удобно складывать}
последовательные или одновременные
\textbf{перемещения}.
\begin{figure}[h]
  \centering
  \begin{tikzpicture}[thick]
    \coordinate (A) at (0cm,0cm);
    \coordinate (B) at (3cm,3cm);
    \coordinate (C) at (7cm,3cm);

    \draw [blue, -Latex, very thick] (A) -- node[above left,black] {\large $\vv{s_1}$} (B);
    \draw [blue, -Latex, very thick] (B) -- node[above,black] {\large $\vv{s_2}$} (C);
    \draw [red, -Latex, very thick] (A) -- node[below right,black] {\large $\vv{s} =\vv{s_1}+\vv{s_2}$} (C);
  \end{tikzpicture}
  \caption{\small Сложение перемещений по правилу треугольника.}\label{pic:sum_perem}
\end{figure}

\clearpage

%%%%%%%%%%%%%%%%%%%%%%%%%%%%%%%%%%%%%%%%%%%%%%%%%%%%%%%%%%%
\subsection{Правило параллелограмма}
%%%%%%%%%%%%%%%%%%%%%%%%%%%%%%%%%%%%%%%%%%%%%%%%%%%%%%%%%%%
Для построения суммы двух векторов $\vv{a}$ и $\vv{b}$
по правилу параллелограмма отметим произвольную точку $A$
и отложим от этой точки вектор $\vv{AB}$, равный $\vv{a}$.
Затем от этой же точки отложим вектор $\vv{AD}$, равный $\vv{b}$.
Вектор $\vv{AC}=\vv{c}$ называется \textbf{суммой векторов} $\vv{a}$ и $\vv{b}$.

Таким образом, при построении суммы двух векторов $\vv{a}$ и $\vv{b}$
начала векторов совмещаются,
и на них как на сторонах строится параллелограмм. Диагональ этого параллелограмма,
проведённая из общего начала складываемых векторов, называется их суммой.
\begin{figure}[h]
  \centering
  \begin{tikzpicture}[thick]
    \coordinate [label=below left:$A$](A) at (0cm,0cm);
    \coordinate [label=above:$B$](B) at (3cm,3cm);
    \coordinate [label=above:$C$](C) at (7cm,3cm);
    \coordinate (D) at (-6cm,0cm);
    \coordinate (E) at (-3cm,3cm);
    \coordinate (F) at (-5cm,4cm);
    \coordinate (G) at (-1cm,4cm);
    \coordinate [label=below:$D$] (H) at (4cm,0cm);

    
    \draw [blue, -Latex, very thick] (A) -- node[above left,black] {\large $\vv{a}$}(B);
    \draw [dashed] (B) -- (C);
    \draw [red, -Latex, very thick] (A) -- (C);
    \draw [blue, -Latex, very thick] (A) -- node[below,black] {\large $\vv{b}$}(H);
    \draw [dashed] (H) -- (C);
    \draw [blue, -Latex, very thick] (D) -- node[above left,black] {\large $\vv{a}$}(E);
    \draw [blue, -Latex, very thick] (F) -- node[above,black] {\large $\vv{b}$}(G);
    
    \draw [fill=black] (A) circle (1.5pt);% node [above] {A};
    
    \draw (8.3cm,2.6cm) node {\large $\vv{c} =\vv{a}+\vv{b}$};
    
  \end{tikzpicture}
  \caption{\small Сложение векторов по правилу параллелограмма.}\label{pic:sum_par}
\end{figure}

По правилу параллелограмма \textbf{удобно складывать}
приложенные к телу
\textbf{силы}.

\begin{figure}[h]
  \centering
  \begin{tikzpicture}[thick, scale = 0.8]
    \coordinate (A) at (0cm,0cm);
    \coordinate (B) at (3cm,3cm);
    \coordinate (C) at (7cm,3cm);
    \coordinate (H) at (4cm,0cm);
   
    \draw [blue, -Latex, very thick] (A) -- node[above left,black]{\large $\vv{F_1}$} (B);
    \draw [dashed] (B) -- (C);
    \draw [red, -Latex, very thick] (A) -- (C);
    \draw [blue, -Latex, very thick] (A) -- node[below,black]{\large $\vv{F_2}$}(H);
    \draw [dashed] (H) -- (C);

    \draw (8cm,3.6cm) node {\large $\vv{F} =\vv{F_1}+\vv{F_2}$};
  \end{tikzpicture}\hspace{1.4cm}
  \begin{tikzpicture}[thick, scale = 0.8]
    \coordinate (A) at (0cm,0cm);
    \coordinate (B) at (-2cm,3cm);
    \coordinate (C) at (2cm,3cm);
    \coordinate (H) at (4cm,0cm);
   
    \draw [blue, -Latex, very thick] (A) -- node[below left,black]{\large $\vv{F_1}$}(B);
    \draw [dashed] (B) -- (C);
    \draw [red, -Latex, very thick] (A) -- (C);
    \draw [blue, -Latex, very thick] (A) -- node[below,black]{\large $\vv{F_2}$}(H);
    \draw [dashed] (H) -- (C);

    \draw (3.2cm,3.6cm) node {\large $\vv{F} =\vv{F_1}+\vv{F_2}$};
  \end{tikzpicture}
  \caption{\small Сложение сил по правилу параллелограмма.}\label{pic:sum_sily}
\end{figure}
